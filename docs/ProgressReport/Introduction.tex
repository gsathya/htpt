\section{Introduction}
Internet censorship is a fact of life in many locations and circumvention tools are the only access some citizens have to unfiltered information. The censor restricts access to these circumvention utilities to maintain information control and a cat and mouse game is born as the circumvention tools try to evade detection. Tor is a useful tool for circumventing Internet censorship that encrypts data and provides an untraceable method for accessing data.

For this project, we have developed a SOCKS proxy that sits below Tor to provide a level of obfuscation such that the data stream is not an trivially detected Tor stream. The initial goal of the project was to develop a Pluggable Transport\cite{Ref1}, but this was found to be not feasible in the timeframe of the project. A Pluggable Transport is an obfuscation layer between a Tor client and a Tor bridge to prevent detection of otherwise  well known Tor data streams. Pluggable Transports consist of two parts: a client and a server. Typically these are slightly different obfuscate/deobfuscate layers, however they can be very different depending on what the obfuscation mechanism is. This obfuscate/deobfuscate layers operates below Tor and relays Tor traffic in a form that is difficult for DPI device to detect in a computationally cost effective manner. In some cases, the Pluggable Transport mechanism mimics or tunnels through an existing protocol\cite{Ref2,Ref3} and in other cases, the transport mechanism tries to appear random\cite{Ref4}. Implementing the obfusate/deobfuscate layers as a SOCKS proxy is a proof of concept, and allows for further development into a proper Pluggable Transport

For our proxy, the obfuscation layer will hide data within HTTP resources in HTTP requests and responses. More specifically, data will be encoded to look like images or more general HTTP traffic and either uploaded to the server or downloaded to the client. By allowing for both upload and download, we are not limiting the application of this pluggable transport to primarily unidirectional traffic. 
